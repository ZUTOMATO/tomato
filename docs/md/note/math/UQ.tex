\documentclass[a4paper,10pt]{article}
\usepackage{titlesec}
\usepackage{ctex}
\usepackage{geometry}
\geometry{left=2cm,right=2cm,top=2.5cm,bottom=2.5cm}
\usepackage{multirow}
\usepackage{cite}
\usepackage{enumitem}
\usepackage{multicol} 
\usepackage{fancyhdr}
\usepackage{amsmath}

\renewenvironment{abstract}{%
    \noindent{\heiti\zihao{5} 摘\hspace{1em}要}%
    \small                              
    \vspace{0.5em}                      
    \noindent\ignorespaces               
}{}

\newenvironment{keywords}{%
    \noindent{\heiti\zihao{5} 关键词}%
    \small
    \vspace{0.5em}
    \noindent\ignorespaces
}{}

\renewcommand{\normalsize}{\fontsize{9pt}{10.8pt}\selectfont}
\titleformat{\section}{\heiti\Large}{\thesection}{1em}{}
\titleformat{\subsection}{\heiti\normalsize}{\thesubsection}{1em}{}
\titleformat{\subsubsection}{\kaishu\small}{\thesubsubsection}{1em}{}

% 设置页眉和页脚
\pagestyle{fancy}
\fancyhf{} % 清除所有页眉和页脚
\lhead{不确定性量化导论} % 右上角小字
\renewcommand{\headrulewidth}{0.5pt} % 设置分割线宽度


\title{\LARGE 不确定性量化导论}
\date{}

\begin{document}

\maketitle
\section{数值差商}
对于微分方程$\begin{cases}
\frac{dy}{dx}=f(x,y)  \\
y(a) = y_0
\end{cases}$
向前差商$\frac{y(x_{n+1}-y(x_{n}))}{\Delta x}\approx y'(x_n)$ 
\section{Runge-Kutta方法(泰勒展开)}
$y'(x)=f(x,y), \quad y''(x)=f_x(x,y)+f_y(x,y)f(x,y),\cdots$
截断$T=O(\Delta x^{p})$,带入可得
$$y^{n+1} = y^{n}+\cdots $$
如
\begin{itemize}
	\item $p=1,\quad y^{n+1}=y^{n}+\Delta x f(x_n,y^{n})$即为欧拉方法;
	\item $p=2,\quad y^{n+1}=y^{n}+\Delta x f(x_n,y^{n})+\frac{\Delta x^{2}}{2}\left[f_x(x,y)+f_y(x,y)f(x,y)\right]$
\end{itemize}
\subsection{3阶Strong Stability Preserving方法}
\section{PDE的有限差分方法}
$$u_t(x,t)=au_x(x,t),\quad x\in [0, 2\pi]
$$$$
u(x,0)=f(x)$$边界是周期的:$$
\frac{d^{p}f}{dx^{p}}(0)=\frac{d^{p}f}{dx^{p}}(2\pi)$$
\end{document}